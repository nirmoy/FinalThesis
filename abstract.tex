\paragraph\
Mobile Adhoc Networks(MANETs) allow for infrastructure-less  mobile wireless networks to be established with minimal delay for deployment. The trend seen in applications of wireless services is that they make increasingly complex demands for Quality of Service(QoS). Thus, providing QoS in a MANET environment is becoming more important. However, QoS in MANETs is a challenging task, because of the inherent MANET limitations and properties. Some of the existing solutions are not scalable and have high signaling overhead. Some solutions are scalable but do not use the resources efficiently. Some solutions are simple but they do not even guarantee QoS and typically do not support multiple classes of service.
\paragraph\
Our proposed scheme is a cross-layer QoS-aware routing protocol that uses Diffserv model for data plane operations. It supports multiple classes of service and dynamically allocates resources for each QoS class. It reserves the resources for each flow and each flow is mapped to one of the QoS classes. The number of meters, policers and queues maintained per node are restricted to number of QoS levels only. So in this way it is scalable. As it reserves resources during route discovery process itself, it has less signaling overhead and has less latency to start data plane operations. To evaluate the performance of our scheme, we implemented it in the network simulator \textit{ns-2.29} by extending Dynamic Source Routing (DSR) protocol.
\paragraph\
Simulation results show that our scheme has achieved throughput close to ASAP while using fewer meters, policers and queues. Results also show that call acceptance ratio of our scheme is higher than ASAP. From the results we can also observe that our scheme gives QoS guarantee to the flows when congestion occurs whereas SWAN does not. Average end-to-end delay for our scheme is less than that of both ASAP and SWAN. Latency to start data plane operations is also acceptable in our scheme. Overall, we are performing better than ASAP and SWAN.
\section{Overview}
\paragraph\
    OpenVswitch, a multilayer software switch that supports NetFlow, sFlow, SPAN, RSPAN, CLI, LACP, 802.1ag as management 
    protocols is been heavily used for network virtualization platforms. In this chapter I describe about design and 
    implementation of OpenVswitch modification to support scalable flow monitoring. We have able to enable writing NetFlow
    data directly to the Cassandra database from OpenVswitch using shared memory and Python Daemon. The last section detail outs our 
    findings of experiments that done on modified OpenVswitch.
    
    \section{NetFlow Monitoring}
    \paragraph\
    NetFlow is the de-facto protocol for flow monitoring. Routers maintains ``flow cache" contains flow records that forwarded
    by the router. These flow records later exported to a NetFlow collector using UDP. Flows are identify by IP address,
    source and destination port, protocol and type of service. When a packet reaches one of its interface, it update ``flow cache"
    if packet belongs to existing flows otherwise it creates a new flow. A flow is ended depending on these rules:
    \begin{enumerate}
     \item When a TCP RST or FIN received.
     \item Active time out is over.
     \item Inactive time out is over.
     \item When ``flow cache" is over.
    \end{enumerate}
    Other than flow key each flow records contains timestamps of first and last packet received, number of packet and bytes  received,
    SNMP index. These flow records fields can be use for multiple analysis, like port numbers can be use for identifying application
    patterns and so on. Table \ref{netflowheader} describes about NetFlow packet header anf table \ref{netflowrecord} describes 
    about NetFlow record format.
    \begin{table}
     \label{netflowheader}
     \begin{tabular}{|p{2cm}|p{11cm}|p{2cm}|}
      \hline
      {\bf Name} & {\bf Description} & {\bf Length in Bytes} \\ \hline
      Version & Type of record format. & 2 \\ \hline
      Count   & Number of flow records contained. & 2 \\ \hline
      SysUpTime & How long the system has been up and running. & 4 \\ \hline
      Epoch	& UNIX timestamp value when the packet was sent. & 4 \\ \hline
      Nanoseconds & Residual nanoseconds after epoch second.	& 4\\ \hline
      Flow Seen	& Total number of flow seen since the exporter began emitting flow detail records. & 4\\ \hline
      Engine Type & User-configurable value (0-255) assigned to the exporter. & 1 \\ \hline
      Engine ID  & User-configurable value (0-255) assigned to the exporter. & 1 \\ \hline
      Sampling Interval & First two bits hold the sampling mode; remaining 14 bits hold value of sampling interval. & 2 \\ \hline
     \end{tabular}
     \caption{Netflow Header Format}
    \end{table}
    \begin{table}
     \begin{tabular}{|p{4cm}|p{9cm}|p{2cm}|}
      \hline
      {\bf Name} & {\bf Description} & {\bf Length in Bytes} \\ \hline
      Source Address & Source IP address. & 4 \\ \hline
      Destination Address & Destination IP address. & 4 \\ \hline
      Next Hop & IP address of next hop router. & 4 \\ \hline
      Input & SNMP index of input interface. & 2 \\ \hline
      Output & SNMP index of output interface & 2 \\ \hline
      Packet Count & Total packets in the flow. & 4 \\ \hline
      Byte Count & Total bytes in teh flow. & 4 \\ \hline
      First & System up time when flow started. & 4 \\ \hline
      Last  & System up time when flow ended. & 4 \\ \hline
      Source Port & TCP/UDP source port. & 2 \\ \hline
      Destination Port & TCP/UDp destination port. & 2 \\ \hline
      Pad1 & Unused byte. & 1 \\ \hline
      TCP Flags & Cumulative OR of TCP flags. & 1 \\ \hline
      Protocol &  IP protocol type (for example, TCP = 6; UDP = 17). & 1 \\ \hline
      TOS & Type of service & 1 \\ \hline
      Source AS & Autonomous system number of the source, either origin or peer. & 4 \\ \hline
      Destination AS & Autonomous system number of the destination, either origin or peer. & 4 \\ \hline
      Source Mask & Source address prefix mask bits. & 4 \\ \hline
      Destination Mask & Destination address prefix mask bits. & 4 \\ \hline
      Pad2 & Unused & 2 \\ \hline
      \end{tabular}
     \label{netflowrecord}
     \caption{NetFlow Record Format}
    \end{table}


    
    \paragraph{OpenVswitch as NetFlow Exporter:} OpenVswitch supports NetFlow protocol. OpenVswitch, a Openflow software switch that 
    maintains basic flow statistics for each OpenFlow rules. Flow key in define by OpenFlow switch is different than 
    flow key in general NetFlow enabled router or switchs. OpenFlow switches maintain ``flow table", an entry in ``flow table"
    contains three fields . These are :
    \begin{enumerate}
     \item A packet header that define flow tuple which can include upto 10 tuple in below.
	  \begin{enumerate}
	   \item In Port.
	   \item VLAN ID.
	   \item Source MAC.
	   \item Destination MAC.
	   \item Source IP.
	   \item Destination IP.
	   \item Ethernet Type.
	   \item IP Protocol.
	   \item Source IP Port.
	   \item Destination IP Port.
	  \end{enumerate}
      \item A action, which defines how the packet should be processed.
      \item Statistics, which keep track of number of packets and bytes, time since last packet seen.
    \end{enumerate}
    Using fields 1 and 3 a OpenFlow enable switches creates NetFlow packets. Though OpenVswitch calculates NetFlow in different way, 
    it improves network visibility to 100\% \cite{sdnnetflow}.
    \paragraph{Limitations of NetFlow Analysis using on Collectors:} 
    NetFlow monitoring can be done either by storing flow records in some database and later analyzed by another application or 
    NetFlow collector itself can collect and analyze flow records and store the analyzed data into database.\emph{ntop} follows these
    analysis-store approach that is one of the reason why \emph{ntop} losses packets on high speed network. Next section describes our 
    proposed solutions to the problem.
    
\section{Proposed Scalable Solution for Flow Monitoring}
\paragraph\
    Currently OpenVswitch can exports NetFlow packets as UDP packets to a  collector like \emph{ntop} that receives NetFlow records then 
    analysis them. UDP is connection less protocol, it does not guarantee reliable communication. Packets may get lost by \emph{ntop}
    while processing because of socket buffer overflow. In previous chapter, our results shows that if \emph{ntop} has less socket buffer 
    it tent to losses packets. Our solution tries to avoid NetFlow collector all together allowing OpenVswitch host to store NetFlow
    records directly to scalable database, Cassandra. Now any NetFlow analysis tool can read flow records from the database and then analysis
    them. As OpenVswitch is distributed in nature writing flow records directly from OpenVswitch host to Cassandra provides scalable
    way to collect flow records. Cassandra is scalable and distributed database offer scalable storage for our solution.
    
    \subsection{Design}
    \paragraph\
	Our solution uses shared memory for storing NetFlow records from OpenVswitch. Another Python program that checks the shared 
	memory region for new flow records store them into Cassandra using $Pycassa$. By this we avoiding any NetFlow collector which can
	become bottleneck. 

    
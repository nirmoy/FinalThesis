\paragraph\
  Flow monitoring protocols like NetFlow \cite{NetFlow} and sFlow \cite{sFlow} can provide   important information about traffic that 
  passes through a network. However, contemporary  networking with its 10Gbps and higher NICs is outpacing the ability to monitor them 
  efficiently. As data centers are getting virtualized with virtual software switches and scaling to thousands  of nodes, 
  it is an immediate requirement to have monitoring systems that can scale effectively. There are not many solutions proposed that 
  provide scalable flow monitoring in data center networks. In this chapter, we review some of the literature that deals with scalable 
  flow monitoring in data center networks.
  
  \section{Edge Monitoring and Collection for Cloud\\ (EMC2) \cite{emc2}} 
  \paragraph\
    Edge Monitoring and Collection for Cloud (EMC2) is a scalable network-wide monitoring service for data centers. 
    EMC2 stays inside the host computer  to monitor virtual switches. Monitoring at virtual switch is scalable due to 
    the distributed nature of the storage of the collected information.
    
  \subsection{Architecture}
    \begin{figure}[htb]
          \centering
          \includegraphics[scale=.35]{emc2.png}
          \caption{Architecture of EMC2 \cite{emc2}.} 
          \label{EMC2_arch}
    \end{figure}
    
    Figure \ref{EMC2_arch} shows the architecture of EMC2.

    \paragraph\
    EMC2 is a multi-threaded application that contains the following modules:
    \begin{enumerate}
     \item Flow-Table : Flow-Table is an in-memory 2-level hash table.
     \item NetFlowParser : It parses the NetFlow datagrams and updates the Flow-Table.
     %AP: ``It'' is a singular pronoun. Should be associated with a singular verb ``parses'' not the plural verb ``parse''. Similarly ``updates'' and not ``update''. They update, it updates, she updaes, he updates and so on. Please use the proper singular noun/pronoun and verb combinations.
     \item sFlowParser : It parses the sFlow datagrams and updates the Flow-Table.
     %AP: Missing articles: you need to say ``the sFlow'' and ``the Flow-Table'' since you are talking about a specific entity. If you are talking in general, then you need to use the article ``a'' or ``an'' depending on whether the word starts with a consonant or a vowel resp.
     \item NetFlowCollector : It accepts the NetFlow datagrams and creates the parser thread upon receiving NetFlow datagrams.
     \item sFlowCollector : It does the same task like the NetFlowCollector but for sFlow datagrams.
     \item FlowCollector : It invokes two thread -- NetFlowCollector and sFlowCollector -- for accepting flow datagrams.
  \end{enumerate}


    \subsubsection{Flow-Table}
      \paragraph\
	Flow-Table is a 2-level in-memory hash table. The primary key for the hash table is the Flow ID which is formed based on the 
	layer 3 source and destination addresses. Flow ID maps to another hash table where timestamp is the key and flow record is the value. Each flow record contains number of packets, number of bytes and optional path vector.
%AP: Overuse of hyphen (-). Please do not hyphenate everything.

    \subsubsection{NetFlowParser/sFlowParser}
      \paragraph\
	NetFlowCollector and sFlowCollector create these two parser threads upon receiving a NetFlow/sFlow datagram respectively. 
	These parser threads parse the datagram and update the Flow-Table. They also perform two important tasks:
	%AP: ``Parser threads'' is plural - hence use the plural verb ``perform'' not the singular ``performs''. Also, do not keep on saying ``Parser threads''. You can say ``They'' after the first one or two sentences. It is obvious that ``they'' refers to the preceding reference whatever that may be.

	\begin{enumerate}
	 \item Deduplication.
	 \item Data rate prediction in presence of sampling.
	\end{enumerate}
   %AP: Use enumerate wherever possible as it gives you numbering. Bullets are used in more special cases such as primarily in slides etc.
   
   \paragraph{Deduplication:}
       Deduplication prevents duplicate flow records from being added to the Flow-Table. 
       %AP: Please see how I have modified the above sentence from what you had written.
       It uses the following algorithm.
       
       
     \begin{algorithm}[H]
        \caption{Detect Duplicate Flow}
	\label{alg1}

	\begin{algorithmic}
	  \IF{$flow-ID$ not exist}

	      \STATE add flow to the flow table.
	      \RETURN 

	  \ELSE 

	      \IF {Same exporter}
		  \STATE update the flow table.
		  \RETURN 

	      \ELSE

	        \STATE report duplicate flow.
	        \STATE update path vector.
	        \RETURN 

	      \ENDIF

	  \ENDIF

	\end{algorithmic}

       \end{algorithm}

    \paragraph{Data Rate Prediction in Presence of Sampling:}
       Sampling rate is specified in flow datagrams. Parser thread predicts the data rate by multiplying  sampling rate with the length 
       of the packet.
   %AP: This is not clear to me still and as I said does not convey what you told me. I will defer modification of this until I have read the paper.
   
    \subsubsection{NetFlowCollector/sFlowCollector}
      NetFlowCollector/sFlowCollector are collector thread that wait %(AP: wait not waits)
      for new NetFlow or sFlow datagrams and spawn a new NetFlowParser/sFlowParser thread upon receiving a datagram.
      
    \subsection{Advantages and Limitations}
    The authors state the following advantages of EMC2:

    \begin{itemize}
     \item Scalable monitoring as EMC2 monitors host $vswitches$ in a distributed fashion by storing the information as flat files in
     those hosts itself instead of sending them to a centralized collector.
     %AP: I hope the above is correct in terms of the content of the paper.
    \end{itemize}
    
    However, flat files are not really built for scalability unlike many other distributed databases available today such as Cassandra, 
    Big Table etc.. Therefore, it is not clear how much of performance can be obtained by storing the information in flat files in the host 
    itself. Considering that most data centers work with SANs rather than local disks, this may not be as scalable as claimed.
    
    %AP: There is a real problem with your references. But, I will look at how you wrote your .bib file etc. on Monday.

   \section{Scalable Internet Traffic Measurement and Analysis with Hadoop \cite{Lee}}
   \paragraph\
   Hadoop \cite{hadoop} is a distributed computing platform that uses a distributed file system(HDFS) and MapReduce \cite{mapreduce} 
   programming model. A Hadoop cluster consists of 
   commodity hardware that can scale  to thousands of nodes to store huge amounts of data. It can perform massive data analytics operations on 
   the available data using MapReduce. Storing NetFlow data on Hadoop and analysis using MapReduce offers scalable traffic 
   measurement and analysis.
   %Nirmoy:add images to describe Hadoop cluster and MapReduce 
   %\subsection{Hadoop Architecture}
    \subsection{Architecture}
     Traffic measurement and analysis system with Hadoop consists of the following modules:
     \begin{enumerate}
      \item Traffic collector.
      \item IP packet and NetFlow reader in Hadoop.
      \item Analysis in MapReduce.
      \item Interactive query interface with Hive \cite{hive}.
     \end{enumerate}

     \subsubsection{Traffic collector}
     \paragraph\
     Traffic collection is done by a high-speed packet capture driver and load balancer. Load balancer forwards packets into
     multiple Hadoop data nodes evenly. Traffic collector also reads trace files stored on the disk and writes them into HDFS. 
     Trace files contains Netflow packets or IP packets in $libpcap$ format.
	
     \subsubsection{IP packet and NetFlow reader in Hadoop}
     \paragraph\
      Storing binary trace files into Hadoop specific sequence files needs more computation power as every packet has to be sequentially 
      read from the trace file and stored into HDFS. The authors have %AP: Authors is a plural, so the plural verb is ``have'' not ``has''.
      developed new Hadoop APIs to store trace files directly into HDFS. 
      As there is no distinct mark to find out the end of a packet record %AP: When you use the article ``a'' it means a single entity - you cannot say ``a packet records''.
      in $libpcap$ format, authors proposed a heuristic algorithm using timestamp-based bit pattern.

      \paragraph{Timestamp-based heuristic algorithm using MapReduce to identify packet records:}
	MapReduce job invokes multiple map tasks to process each	 HDFS block in parallel. 
	%AP: Do you mean that one block is processed in parallel or multiple blocks can be processed in parallel? If the latter, we have to change the above sentence? 
	Each of the map tasks follows these steps to identify the packet records in a HDFS block:
	\begin{enumerate}
	 \item Read two records using the $libpcap$ 16-byte header.
	 \item Check timestamp value, that should be within duration of captured time.
	 %AP: Cannot understand what you mean
	 \item Difference between wired packet length and captured length should be less than maximum packet length.
	 \item Check whether timestamp difference between the two packet records is within the defined
	 %AP: not ``within the define'' - you have to use past tense here.
	 threshold. 
	\end{enumerate}
	%AP: There is no way I can understand what the heuristic is and how it is helping from your description of the algorithm here. We will have to discuss this and I may have to re-write it to make a coherent algorithm.
	
	\subsubsection{Analysis in MapReduce}
	The authors have %AP: again, authors is plural and so you must use ``have'' - also, you must use the definite article ``the'' as the authors are defined - not just any authors in the world.
	implemented  analysis tools for processing IP packet as well as NetFlow packets using MapReduce algorithms. The tools implemented are:
	
	\begin{enumerate}
	 \item IP Layer analysis tools provide the following analysis jobs:
	      \begin{enumerate}
	       \item Host and port count statistics.
	       \item Periodic flow statistics.
	       \item Periodic sample traffic statistics.
	      \end{enumerate}
	 \item TCP layer analysis computes the following statistics:
	       \begin{enumerate}
	        \item RTT.
	        \item Retransmission.
	        \item Throughput.
	       \end{enumerate}
	\item NetFlow analysis provides 
	      \begin{enumerate}
	       \item Human readable flow statistics.
	       \item Aggregated flow statistics.
	       \item Top flows sorted by a key such as packet count or byte count. 
	      \end{enumerate}
	\end{enumerate}

      \subsubsection{Interactive query interface with Hive}
      Hive is a data warehousing system build on top of Hadoop that allows us to generate MapReduce jobs using SQL like query.
      IP analysis MapReduce jobs process NetFlow packets on HDFS and  store flow record and IP statistics into
      Hive tables. A user can then query the Hive tables using the interactive web-based user interface.
      
      \subsection{Advantages and Limitations}
      Advantages of Hadoop based traffic  measurement and analysis are 
	\begin{enumerate}
	  \item Scalable storage.
	  \item MapReduce operations on flow data.
	\end{enumerate}
      Disadvantages of Hadoop based traffic  measurement and analysis are
	\begin{enumerate}
	 \item Low response time- ``the fastest MapReduce job takes 15+ seconds" \cite{ha}.
	 \item Hadoop NameNode was a single point of failure which was solved in later version of 
	        Hadoop 2.0.0 with passive NameNodes.
	 \item Multiple NameNodes required to get high availability \cite{ha}.
	\end{enumerate}

	
      \section{\emph{nfdump} \cite{nfdump}}
      \paragraph\
      \emph{nfdump} provides a set of tools to capture and analyse NetFlow packets. The set of tools are :
      \begin{enumerate}
	\item \emph{nfcapd}
	\item \emph{nfdump}
	\item NfSen 
       \end{enumerate}

       \emph{nfcapd} reads data from the network and stores them into the disk. It also rotates %AP: rotates - remember the verb plural/singular rules - also, I am not anyway happy with the word rotate - dont you thnk CS has nice technical terms for such a behavior? Such as ''It treats the file as a circular queue to limit the size of the file``.
       the file to limit size of file.
      \emph{nfdump} allows $tcpdump$ style filter expression for processing NetFlow data stored by \emph{nfcapd} and displaying them on terminal or writing them into a file. NfSen gives a graphical overview of NetFlow data using RRDTool \cite{rrd};
        \begin{figure}[htb]
          \centering
          \includegraphics[scale=.35]{nfdump.png}
          \caption{Architecture of \emph{nfdump}.} 
	\end{figure}

	\subsection{Advantages and Limitations}
      \emph{nfdump} is suitable for small scale NetFlow analysis. In case of large scale NetFlow monitoring disk file based storage is not sufficient.
	
     \section{Packet Forwarding with \emph{netmap}\cite{spf}}
     \paragraph\
     This section describes about performance shortfalls of OpenVswitch \cite{ovs}, a study done by Luigi Rizzo and his team.
     \emph{netmap} \cite{netmap} is a framework to reduce the cost of moving traffic between the hardware and the host stack. As our network
     infrastructures moving from Fast Ethernet to Gigabit Ethernet and host stack APIs are not fast enough to handle such speed, 
     framework like \emph{netmap} is essential for such environment. %\emph{netmap} implements zero copy packet forwarding using shared memory
    % and data structures tune for fast packet receiving/forwarding.
    \paragraph\
    \emph{netmap} achieves efficient I/O by removing all unnecessary run time costs and system calls. \emph{netmap} exposes its API to userspace
    processes and provides a shadow copies of NIC rings.  Memory regions containing \emph{netmap} rings and packet buffers is shared with a
    process using $mmap$(). A user process can access NIC ring buffers directly, so transferring data can be done with zero copy operations.
    \emph{netmap} has a small \emph{libpcap} compatible API for migrating existing application.% to migrate to \emph{netmap}
     \paragraph\
     OpenVswitch is a multi-layer, OpenFlow \cite{openflow} compliant virtual switch designed to automate network through programmatic extensions. OpenVswitch is currently 
     heavily used for network virtualization. So performance analysis of OpenVswitch is critical as we are modifying it to support
     scalable flow monitoring.
     
     Authors have used \emph{netmap} for \emph{libpcap} alternative to improve the performance of OpenVswitch from 65 Kpps to 1.3 Mpps.
     After resolving few architectural limitations of OpenVswitch, authors have able to get three times performance improvement.
     Table \ref{openvswitch-perf} shows the performance of OpenVswitch under various environment.\\
     \begin{table}[th]   
      \centering
      \begin{tabular}{|l|r|}
        \hline
        {\bf Configuration} & {\bf Kpps}  \\ \hline      
         Linux userspace     & 50          \\ \hline 
         FreeBSD userspace   & 65          \\ \hline 
         Linux Kernel        & 300         \\ \hline 
         Optimize OpenVswitch + FreeBSD & 790  \\ \hline
         Optimize OpenVswitch + FreeBSD + \emph{netmap} & 3050  \\ \hline
      
         \end{tabular}
       \caption{OpenVswitch Performance with \emph{netmap}}
       \label{openvswitch-perf}
     \end{table}
     So from this table we can easily find impressive performance gain by OpenVswitch using \emph{netmap}.
     This paper \cite{spf} also gives us internal architecture of OpenVswitch that helped to modify OpenVswitch
     for our second approach to have scalable flow monitoring with $Vswitch$.
     
     
     %\section{Virtual I/O for Virtual Machines with \emph{virtio} \cite{virtio}}
     %\paragraph\
      
     
     
     
      \section{Summary}
      In this section I have discussed  available solutions for scalable flow monitoring. The issues with current solutions are 
      \begin{enumerate}
       \item Lack of scalable storage \cite{emc2} \cite{nfdump}.
       \item Lack of high availability data storage \cite{Lee}.
       \item Real-time flow analysis is difficult with current systems \cite{Lee}.  
      \end{enumerate}

      
      In the next chapter I describe the design and implementation of modified \emph{ntop}(a NetFlow collector) that 
      tries to solve a few of these issues.
      
